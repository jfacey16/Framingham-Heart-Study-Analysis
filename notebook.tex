
% Default to the notebook output style

    


% Inherit from the specified cell style.




    
\documentclass[11pt]{article}

    
    
    \usepackage[T1]{fontenc}
    % Nicer default font (+ math font) than Computer Modern for most use cases
    \usepackage{mathpazo}

    % Basic figure setup, for now with no caption control since it's done
    % automatically by Pandoc (which extracts ![](path) syntax from Markdown).
    \usepackage{graphicx}
    % We will generate all images so they have a width \maxwidth. This means
    % that they will get their normal width if they fit onto the page, but
    % are scaled down if they would overflow the margins.
    \makeatletter
    \def\maxwidth{\ifdim\Gin@nat@width>\linewidth\linewidth
    \else\Gin@nat@width\fi}
    \makeatother
    \let\Oldincludegraphics\includegraphics
    % Set max figure width to be 80% of text width, for now hardcoded.
    \renewcommand{\includegraphics}[1]{\Oldincludegraphics[width=.8\maxwidth]{#1}}
    % Ensure that by default, figures have no caption (until we provide a
    % proper Figure object with a Caption API and a way to capture that
    % in the conversion process - todo).
    \usepackage{caption}
    \DeclareCaptionLabelFormat{nolabel}{}
    \captionsetup{labelformat=nolabel}

    \usepackage{adjustbox} % Used to constrain images to a maximum size 
    \usepackage{xcolor} % Allow colors to be defined
    \usepackage{enumerate} % Needed for markdown enumerations to work
    \usepackage{geometry} % Used to adjust the document margins
    \usepackage{amsmath} % Equations
    \usepackage{amssymb} % Equations
    \usepackage{textcomp} % defines textquotesingle
    % Hack from http://tex.stackexchange.com/a/47451/13684:
    \AtBeginDocument{%
        \def\PYZsq{\textquotesingle}% Upright quotes in Pygmentized code
    }
    \usepackage{upquote} % Upright quotes for verbatim code
    \usepackage{eurosym} % defines \euro
    \usepackage[mathletters]{ucs} % Extended unicode (utf-8) support
    \usepackage[utf8x]{inputenc} % Allow utf-8 characters in the tex document
    \usepackage{fancyvrb} % verbatim replacement that allows latex
    \usepackage{grffile} % extends the file name processing of package graphics 
                         % to support a larger range 
    % The hyperref package gives us a pdf with properly built
    % internal navigation ('pdf bookmarks' for the table of contents,
    % internal cross-reference links, web links for URLs, etc.)
    \usepackage{hyperref}
    \usepackage{longtable} % longtable support required by pandoc >1.10
    \usepackage{booktabs}  % table support for pandoc > 1.12.2
    \usepackage[inline]{enumitem} % IRkernel/repr support (it uses the enumerate* environment)
    \usepackage[normalem]{ulem} % ulem is needed to support strikethroughs (\sout)
                                % normalem makes italics be italics, not underlines
    

    
    
    % Colors for the hyperref package
    \definecolor{urlcolor}{rgb}{0,.145,.698}
    \definecolor{linkcolor}{rgb}{.71,0.21,0.01}
    \definecolor{citecolor}{rgb}{.12,.54,.11}

    % ANSI colors
    \definecolor{ansi-black}{HTML}{3E424D}
    \definecolor{ansi-black-intense}{HTML}{282C36}
    \definecolor{ansi-red}{HTML}{E75C58}
    \definecolor{ansi-red-intense}{HTML}{B22B31}
    \definecolor{ansi-green}{HTML}{00A250}
    \definecolor{ansi-green-intense}{HTML}{007427}
    \definecolor{ansi-yellow}{HTML}{DDB62B}
    \definecolor{ansi-yellow-intense}{HTML}{B27D12}
    \definecolor{ansi-blue}{HTML}{208FFB}
    \definecolor{ansi-blue-intense}{HTML}{0065CA}
    \definecolor{ansi-magenta}{HTML}{D160C4}
    \definecolor{ansi-magenta-intense}{HTML}{A03196}
    \definecolor{ansi-cyan}{HTML}{60C6C8}
    \definecolor{ansi-cyan-intense}{HTML}{258F8F}
    \definecolor{ansi-white}{HTML}{C5C1B4}
    \definecolor{ansi-white-intense}{HTML}{A1A6B2}

    % commands and environments needed by pandoc snippets
    % extracted from the output of `pandoc -s`
    \providecommand{\tightlist}{%
      \setlength{\itemsep}{0pt}\setlength{\parskip}{0pt}}
    \DefineVerbatimEnvironment{Highlighting}{Verbatim}{commandchars=\\\{\}}
    % Add ',fontsize=\small' for more characters per line
    \newenvironment{Shaded}{}{}
    \newcommand{\KeywordTok}[1]{\textcolor[rgb]{0.00,0.44,0.13}{\textbf{{#1}}}}
    \newcommand{\DataTypeTok}[1]{\textcolor[rgb]{0.56,0.13,0.00}{{#1}}}
    \newcommand{\DecValTok}[1]{\textcolor[rgb]{0.25,0.63,0.44}{{#1}}}
    \newcommand{\BaseNTok}[1]{\textcolor[rgb]{0.25,0.63,0.44}{{#1}}}
    \newcommand{\FloatTok}[1]{\textcolor[rgb]{0.25,0.63,0.44}{{#1}}}
    \newcommand{\CharTok}[1]{\textcolor[rgb]{0.25,0.44,0.63}{{#1}}}
    \newcommand{\StringTok}[1]{\textcolor[rgb]{0.25,0.44,0.63}{{#1}}}
    \newcommand{\CommentTok}[1]{\textcolor[rgb]{0.38,0.63,0.69}{\textit{{#1}}}}
    \newcommand{\OtherTok}[1]{\textcolor[rgb]{0.00,0.44,0.13}{{#1}}}
    \newcommand{\AlertTok}[1]{\textcolor[rgb]{1.00,0.00,0.00}{\textbf{{#1}}}}
    \newcommand{\FunctionTok}[1]{\textcolor[rgb]{0.02,0.16,0.49}{{#1}}}
    \newcommand{\RegionMarkerTok}[1]{{#1}}
    \newcommand{\ErrorTok}[1]{\textcolor[rgb]{1.00,0.00,0.00}{\textbf{{#1}}}}
    \newcommand{\NormalTok}[1]{{#1}}
    
    % Additional commands for more recent versions of Pandoc
    \newcommand{\ConstantTok}[1]{\textcolor[rgb]{0.53,0.00,0.00}{{#1}}}
    \newcommand{\SpecialCharTok}[1]{\textcolor[rgb]{0.25,0.44,0.63}{{#1}}}
    \newcommand{\VerbatimStringTok}[1]{\textcolor[rgb]{0.25,0.44,0.63}{{#1}}}
    \newcommand{\SpecialStringTok}[1]{\textcolor[rgb]{0.73,0.40,0.53}{{#1}}}
    \newcommand{\ImportTok}[1]{{#1}}
    \newcommand{\DocumentationTok}[1]{\textcolor[rgb]{0.73,0.13,0.13}{\textit{{#1}}}}
    \newcommand{\AnnotationTok}[1]{\textcolor[rgb]{0.38,0.63,0.69}{\textbf{\textit{{#1}}}}}
    \newcommand{\CommentVarTok}[1]{\textcolor[rgb]{0.38,0.63,0.69}{\textbf{\textit{{#1}}}}}
    \newcommand{\VariableTok}[1]{\textcolor[rgb]{0.10,0.09,0.49}{{#1}}}
    \newcommand{\ControlFlowTok}[1]{\textcolor[rgb]{0.00,0.44,0.13}{\textbf{{#1}}}}
    \newcommand{\OperatorTok}[1]{\textcolor[rgb]{0.40,0.40,0.40}{{#1}}}
    \newcommand{\BuiltInTok}[1]{{#1}}
    \newcommand{\ExtensionTok}[1]{{#1}}
    \newcommand{\PreprocessorTok}[1]{\textcolor[rgb]{0.74,0.48,0.00}{{#1}}}
    \newcommand{\AttributeTok}[1]{\textcolor[rgb]{0.49,0.56,0.16}{{#1}}}
    \newcommand{\InformationTok}[1]{\textcolor[rgb]{0.38,0.63,0.69}{\textbf{\textit{{#1}}}}}
    \newcommand{\WarningTok}[1]{\textcolor[rgb]{0.38,0.63,0.69}{\textbf{\textit{{#1}}}}}
    
    
    % Define a nice break command that doesn't care if a line doesn't already
    % exist.
    \def\br{\hspace*{\fill} \\* }
    % Math Jax compatability definitions
    \def\gt{>}
    \def\lt{<}
    % Document parameters
    \title{analysis\_6\_writeup}
    
    
    

    % Pygments definitions
    
\makeatletter
\def\PY@reset{\let\PY@it=\relax \let\PY@bf=\relax%
    \let\PY@ul=\relax \let\PY@tc=\relax%
    \let\PY@bc=\relax \let\PY@ff=\relax}
\def\PY@tok#1{\csname PY@tok@#1\endcsname}
\def\PY@toks#1+{\ifx\relax#1\empty\else%
    \PY@tok{#1}\expandafter\PY@toks\fi}
\def\PY@do#1{\PY@bc{\PY@tc{\PY@ul{%
    \PY@it{\PY@bf{\PY@ff{#1}}}}}}}
\def\PY#1#2{\PY@reset\PY@toks#1+\relax+\PY@do{#2}}

\expandafter\def\csname PY@tok@w\endcsname{\def\PY@tc##1{\textcolor[rgb]{0.73,0.73,0.73}{##1}}}
\expandafter\def\csname PY@tok@c\endcsname{\let\PY@it=\textit\def\PY@tc##1{\textcolor[rgb]{0.25,0.50,0.50}{##1}}}
\expandafter\def\csname PY@tok@cp\endcsname{\def\PY@tc##1{\textcolor[rgb]{0.74,0.48,0.00}{##1}}}
\expandafter\def\csname PY@tok@k\endcsname{\let\PY@bf=\textbf\def\PY@tc##1{\textcolor[rgb]{0.00,0.50,0.00}{##1}}}
\expandafter\def\csname PY@tok@kp\endcsname{\def\PY@tc##1{\textcolor[rgb]{0.00,0.50,0.00}{##1}}}
\expandafter\def\csname PY@tok@kt\endcsname{\def\PY@tc##1{\textcolor[rgb]{0.69,0.00,0.25}{##1}}}
\expandafter\def\csname PY@tok@o\endcsname{\def\PY@tc##1{\textcolor[rgb]{0.40,0.40,0.40}{##1}}}
\expandafter\def\csname PY@tok@ow\endcsname{\let\PY@bf=\textbf\def\PY@tc##1{\textcolor[rgb]{0.67,0.13,1.00}{##1}}}
\expandafter\def\csname PY@tok@nb\endcsname{\def\PY@tc##1{\textcolor[rgb]{0.00,0.50,0.00}{##1}}}
\expandafter\def\csname PY@tok@nf\endcsname{\def\PY@tc##1{\textcolor[rgb]{0.00,0.00,1.00}{##1}}}
\expandafter\def\csname PY@tok@nc\endcsname{\let\PY@bf=\textbf\def\PY@tc##1{\textcolor[rgb]{0.00,0.00,1.00}{##1}}}
\expandafter\def\csname PY@tok@nn\endcsname{\let\PY@bf=\textbf\def\PY@tc##1{\textcolor[rgb]{0.00,0.00,1.00}{##1}}}
\expandafter\def\csname PY@tok@ne\endcsname{\let\PY@bf=\textbf\def\PY@tc##1{\textcolor[rgb]{0.82,0.25,0.23}{##1}}}
\expandafter\def\csname PY@tok@nv\endcsname{\def\PY@tc##1{\textcolor[rgb]{0.10,0.09,0.49}{##1}}}
\expandafter\def\csname PY@tok@no\endcsname{\def\PY@tc##1{\textcolor[rgb]{0.53,0.00,0.00}{##1}}}
\expandafter\def\csname PY@tok@nl\endcsname{\def\PY@tc##1{\textcolor[rgb]{0.63,0.63,0.00}{##1}}}
\expandafter\def\csname PY@tok@ni\endcsname{\let\PY@bf=\textbf\def\PY@tc##1{\textcolor[rgb]{0.60,0.60,0.60}{##1}}}
\expandafter\def\csname PY@tok@na\endcsname{\def\PY@tc##1{\textcolor[rgb]{0.49,0.56,0.16}{##1}}}
\expandafter\def\csname PY@tok@nt\endcsname{\let\PY@bf=\textbf\def\PY@tc##1{\textcolor[rgb]{0.00,0.50,0.00}{##1}}}
\expandafter\def\csname PY@tok@nd\endcsname{\def\PY@tc##1{\textcolor[rgb]{0.67,0.13,1.00}{##1}}}
\expandafter\def\csname PY@tok@s\endcsname{\def\PY@tc##1{\textcolor[rgb]{0.73,0.13,0.13}{##1}}}
\expandafter\def\csname PY@tok@sd\endcsname{\let\PY@it=\textit\def\PY@tc##1{\textcolor[rgb]{0.73,0.13,0.13}{##1}}}
\expandafter\def\csname PY@tok@si\endcsname{\let\PY@bf=\textbf\def\PY@tc##1{\textcolor[rgb]{0.73,0.40,0.53}{##1}}}
\expandafter\def\csname PY@tok@se\endcsname{\let\PY@bf=\textbf\def\PY@tc##1{\textcolor[rgb]{0.73,0.40,0.13}{##1}}}
\expandafter\def\csname PY@tok@sr\endcsname{\def\PY@tc##1{\textcolor[rgb]{0.73,0.40,0.53}{##1}}}
\expandafter\def\csname PY@tok@ss\endcsname{\def\PY@tc##1{\textcolor[rgb]{0.10,0.09,0.49}{##1}}}
\expandafter\def\csname PY@tok@sx\endcsname{\def\PY@tc##1{\textcolor[rgb]{0.00,0.50,0.00}{##1}}}
\expandafter\def\csname PY@tok@m\endcsname{\def\PY@tc##1{\textcolor[rgb]{0.40,0.40,0.40}{##1}}}
\expandafter\def\csname PY@tok@gh\endcsname{\let\PY@bf=\textbf\def\PY@tc##1{\textcolor[rgb]{0.00,0.00,0.50}{##1}}}
\expandafter\def\csname PY@tok@gu\endcsname{\let\PY@bf=\textbf\def\PY@tc##1{\textcolor[rgb]{0.50,0.00,0.50}{##1}}}
\expandafter\def\csname PY@tok@gd\endcsname{\def\PY@tc##1{\textcolor[rgb]{0.63,0.00,0.00}{##1}}}
\expandafter\def\csname PY@tok@gi\endcsname{\def\PY@tc##1{\textcolor[rgb]{0.00,0.63,0.00}{##1}}}
\expandafter\def\csname PY@tok@gr\endcsname{\def\PY@tc##1{\textcolor[rgb]{1.00,0.00,0.00}{##1}}}
\expandafter\def\csname PY@tok@ge\endcsname{\let\PY@it=\textit}
\expandafter\def\csname PY@tok@gs\endcsname{\let\PY@bf=\textbf}
\expandafter\def\csname PY@tok@gp\endcsname{\let\PY@bf=\textbf\def\PY@tc##1{\textcolor[rgb]{0.00,0.00,0.50}{##1}}}
\expandafter\def\csname PY@tok@go\endcsname{\def\PY@tc##1{\textcolor[rgb]{0.53,0.53,0.53}{##1}}}
\expandafter\def\csname PY@tok@gt\endcsname{\def\PY@tc##1{\textcolor[rgb]{0.00,0.27,0.87}{##1}}}
\expandafter\def\csname PY@tok@err\endcsname{\def\PY@bc##1{\setlength{\fboxsep}{0pt}\fcolorbox[rgb]{1.00,0.00,0.00}{1,1,1}{\strut ##1}}}
\expandafter\def\csname PY@tok@kc\endcsname{\let\PY@bf=\textbf\def\PY@tc##1{\textcolor[rgb]{0.00,0.50,0.00}{##1}}}
\expandafter\def\csname PY@tok@kd\endcsname{\let\PY@bf=\textbf\def\PY@tc##1{\textcolor[rgb]{0.00,0.50,0.00}{##1}}}
\expandafter\def\csname PY@tok@kn\endcsname{\let\PY@bf=\textbf\def\PY@tc##1{\textcolor[rgb]{0.00,0.50,0.00}{##1}}}
\expandafter\def\csname PY@tok@kr\endcsname{\let\PY@bf=\textbf\def\PY@tc##1{\textcolor[rgb]{0.00,0.50,0.00}{##1}}}
\expandafter\def\csname PY@tok@bp\endcsname{\def\PY@tc##1{\textcolor[rgb]{0.00,0.50,0.00}{##1}}}
\expandafter\def\csname PY@tok@fm\endcsname{\def\PY@tc##1{\textcolor[rgb]{0.00,0.00,1.00}{##1}}}
\expandafter\def\csname PY@tok@vc\endcsname{\def\PY@tc##1{\textcolor[rgb]{0.10,0.09,0.49}{##1}}}
\expandafter\def\csname PY@tok@vg\endcsname{\def\PY@tc##1{\textcolor[rgb]{0.10,0.09,0.49}{##1}}}
\expandafter\def\csname PY@tok@vi\endcsname{\def\PY@tc##1{\textcolor[rgb]{0.10,0.09,0.49}{##1}}}
\expandafter\def\csname PY@tok@vm\endcsname{\def\PY@tc##1{\textcolor[rgb]{0.10,0.09,0.49}{##1}}}
\expandafter\def\csname PY@tok@sa\endcsname{\def\PY@tc##1{\textcolor[rgb]{0.73,0.13,0.13}{##1}}}
\expandafter\def\csname PY@tok@sb\endcsname{\def\PY@tc##1{\textcolor[rgb]{0.73,0.13,0.13}{##1}}}
\expandafter\def\csname PY@tok@sc\endcsname{\def\PY@tc##1{\textcolor[rgb]{0.73,0.13,0.13}{##1}}}
\expandafter\def\csname PY@tok@dl\endcsname{\def\PY@tc##1{\textcolor[rgb]{0.73,0.13,0.13}{##1}}}
\expandafter\def\csname PY@tok@s2\endcsname{\def\PY@tc##1{\textcolor[rgb]{0.73,0.13,0.13}{##1}}}
\expandafter\def\csname PY@tok@sh\endcsname{\def\PY@tc##1{\textcolor[rgb]{0.73,0.13,0.13}{##1}}}
\expandafter\def\csname PY@tok@s1\endcsname{\def\PY@tc##1{\textcolor[rgb]{0.73,0.13,0.13}{##1}}}
\expandafter\def\csname PY@tok@mb\endcsname{\def\PY@tc##1{\textcolor[rgb]{0.40,0.40,0.40}{##1}}}
\expandafter\def\csname PY@tok@mf\endcsname{\def\PY@tc##1{\textcolor[rgb]{0.40,0.40,0.40}{##1}}}
\expandafter\def\csname PY@tok@mh\endcsname{\def\PY@tc##1{\textcolor[rgb]{0.40,0.40,0.40}{##1}}}
\expandafter\def\csname PY@tok@mi\endcsname{\def\PY@tc##1{\textcolor[rgb]{0.40,0.40,0.40}{##1}}}
\expandafter\def\csname PY@tok@il\endcsname{\def\PY@tc##1{\textcolor[rgb]{0.40,0.40,0.40}{##1}}}
\expandafter\def\csname PY@tok@mo\endcsname{\def\PY@tc##1{\textcolor[rgb]{0.40,0.40,0.40}{##1}}}
\expandafter\def\csname PY@tok@ch\endcsname{\let\PY@it=\textit\def\PY@tc##1{\textcolor[rgb]{0.25,0.50,0.50}{##1}}}
\expandafter\def\csname PY@tok@cm\endcsname{\let\PY@it=\textit\def\PY@tc##1{\textcolor[rgb]{0.25,0.50,0.50}{##1}}}
\expandafter\def\csname PY@tok@cpf\endcsname{\let\PY@it=\textit\def\PY@tc##1{\textcolor[rgb]{0.25,0.50,0.50}{##1}}}
\expandafter\def\csname PY@tok@c1\endcsname{\let\PY@it=\textit\def\PY@tc##1{\textcolor[rgb]{0.25,0.50,0.50}{##1}}}
\expandafter\def\csname PY@tok@cs\endcsname{\let\PY@it=\textit\def\PY@tc##1{\textcolor[rgb]{0.25,0.50,0.50}{##1}}}

\def\PYZbs{\char`\\}
\def\PYZus{\char`\_}
\def\PYZob{\char`\{}
\def\PYZcb{\char`\}}
\def\PYZca{\char`\^}
\def\PYZam{\char`\&}
\def\PYZlt{\char`\<}
\def\PYZgt{\char`\>}
\def\PYZsh{\char`\#}
\def\PYZpc{\char`\%}
\def\PYZdl{\char`\$}
\def\PYZhy{\char`\-}
\def\PYZsq{\char`\'}
\def\PYZdq{\char`\"}
\def\PYZti{\char`\~}
% for compatibility with earlier versions
\def\PYZat{@}
\def\PYZlb{[}
\def\PYZrb{]}
\makeatother


    % Exact colors from NB
    \definecolor{incolor}{rgb}{0.0, 0.0, 0.5}
    \definecolor{outcolor}{rgb}{0.545, 0.0, 0.0}



    
    % Prevent overflowing lines due to hard-to-break entities
    \sloppy 
    % Setup hyperref package
    \hypersetup{
      breaklinks=true,  % so long urls are correctly broken across lines
      colorlinks=true,
      urlcolor=urlcolor,
      linkcolor=linkcolor,
      citecolor=citecolor,
      }
    % Slightly bigger margins than the latex defaults
    
    \geometry{verbose,tmargin=1in,bmargin=1in,lmargin=1in,rmargin=1in}
    
    

    \begin{document}
    
    
    \maketitle
    
    

    
    \hypertarget{data-analysis-6}{%
\section{Data Analysis 6}\label{data-analysis-6}}

    \hypertarget{dataset-description}{%
\subsection{Dataset Description}\label{dataset-description}}

    We are analyzing a subset of data from the Framingham Heart Study. The
dataset contains health data for patients, which has been anonymized for
patient confidentiality. The dataset includes records on 1500
participants over a period of time for a set of variables. Data was
collected for participants over 4607 days. The features for this dataset
are \textbf{SEX}, \textbf{CURSMOKE}, \textbf{DIABETES},
\textbf{PREVHYP}, \textbf{EDUC}, \textbf{AGE}, \textbf{TOTCHOL},
\textbf{SYSBP}, \textbf{DIABP}, \textbf{CIGPDAY}, \textbf{BMI},
\textbf{HEARTRTE}, \textbf{GLUCOSE}, \textbf{PERIOD}, and \textbf{TIME}.
\textbf{SEX} is a binary categorical variable for the participants
biological sex, with 1 for men and 2 for women. \textbf{CURSMOKE} is
binary categorical variable for whether the participant smokes or not,
with 0 for no and 1 for yes. \textbf{DIABETES} is binary categorical
variable for whether the participant has diabetes or not, with 0 for no
and 1 for yes. \textbf{PREVHYP} is the response variable and is a binary
categorical variable representing whether the participant has
hypertension or not. \textbf{EDUC} is an ordinal variable for education
level with 4 levels; 1 for 0-11 years of school, 2 for high school
diploma or GED, 3 for some college or vocational school, and 4 is
college degree and more. \textbf{AGE} is a continuous variable for age.
\textbf{TOTCHOL} is a continuous variable for serum total cholesterol.
\textbf{SYSBP} is a continuous variable for systolic blood pressure.
\textbf{DIABP} is a continuous variable for diastolic blood pressure.
\textbf{CIGPDAY} is a continuous variable representing the number of
cigarettes the participant smokes per day. \textbf{BMI} is a continuous
variable representing the patients body mass index. \textbf{HEARTRTE} is
a continuous variable representing heart rate in beats per minute.
Finally, \textbf{GLUCOSE} is a continuous variable that represents
causal serum glucose. The additional variables in this analysis are
\textbf{PERIOD} and \textbf{TIME}. \textbf{PERIOD} is a categorical
variable that buckets time into 3 categories. \textbf{TIME} is a numeric
variable representing the number of days since the baseline exam.

    \hypertarget{data-cleaning}{%
\subsection{Data Cleaning}\label{data-cleaning}}

    In this case, there was no missing data. Therefore, the only data
cleaning to do is to convert categorical variables to factors.

    \hypertarget{exploratory-data-analysis-tests}{%
\subsection{Exploratory Data Analysis
(tests)}\label{exploratory-data-analysis-tests}}

    Next, a few chi squared tests were run to to test if there is a
relationship between categorical variables and the response.
\textbf{PERIOD} did not seem worth checking, however, \textbf{SEX},
\textbf{CURSMOKE}, \textbf{DIABETES}, and \textbf{EDUC}. \textbf{SEX}
and \textbf{EDUC} do not change overtime, therefore, it is irrelevant
that this is time series data for the chi squared tests. The tests still
apply the same. For \textbf{CURSMOKE} and \textbf{DIABETES}, values do
change over time for participants, however, I would still argue the test
have power and are relevant. If a participant has their
\textbf{CURSMOKE} or \textbf{DIABETES} value change over time, they can
also have the response \textbf{PREVHYP} change as well. Therefore, these
tests still show relationships between these variables and the response.
P values ranged from .08 for \textbf{SEX} to 1.194e-09 for
\textbf{CURSMOKE}. All categoricals besides \textbf{SEX} had p values
below .05. We will consider removing \textbf{SEX} from the final model,
however, a p value of .08 suggests that there still is a strong case
that \textbf{SEX} is not independent of \textbf{PREVHYP}. The
contingency tables for these four categoricals are displayed below.

    \hypertarget{table-1-sex}{%
\paragraph{Table 1: SEX}\label{table-1-sex}}

    \begin{longtable}[]{@{}ccc@{}}
\toprule
~ & free of disease & prevalent disease\tabularnewline
\midrule
\endhead
\textbf{Male} & 366 & 336\tabularnewline
\textbf{Female} & 453 & 345\tabularnewline
\bottomrule
\end{longtable}

    \hypertarget{table-2-cursmoke}{%
\paragraph{Table 2: CURSMOKE}\label{table-2-cursmoke}}

    \begin{longtable}[]{@{}ccc@{}}
\toprule
~ & free of disease & prevalent disease\tabularnewline
\midrule
\endhead
\textbf{Yes} & 420 & 456\tabularnewline
\textbf{No} & 399 & 225\tabularnewline
\bottomrule
\end{longtable}

    \hypertarget{table-3-diabetes}{%
\paragraph{Table 3: DIABETES}\label{table-3-diabetes}}

    \begin{longtable}[]{@{}ccc@{}}
\toprule
~ & free of disease & prevalent disease\tabularnewline
\midrule
\endhead
\textbf{Not Diabetic} & 811 & 634\tabularnewline
\textbf{Diabetic} & 8 & 47\tabularnewline
\bottomrule
\end{longtable}

    \hypertarget{table-4-educ}{%
\paragraph{Table 4: EDUC}\label{table-4-educ}}

    \begin{longtable}[]{@{}ccc@{}}
\toprule
~ & free of disease & prevalent disease\tabularnewline
\midrule
\endhead
\textbf{0-11 years} & 287 & 295\tabularnewline
\textbf{High School Diploma or GED} & 274 & 212\tabularnewline
\textbf{Some College or Vocational School} & 153 & 96\tabularnewline
\textbf{College degree or more} & 105 & 78\tabularnewline
\bottomrule
\end{longtable}

    \hypertarget{exploratory-data-analysis-plots}{%
\subsection{Exploratory Data Analysis
(plots)}\label{exploratory-data-analysis-plots}}

    Next, exploratory plots were generated for the continuous variables. For
each, there are overall density plots, seperated by the response value.
This is used to see if there are differences in the distributions of
each continuous variable for the two classes. These charts do not take
into account time, therefore individual trajectories for each continuous
variable for each participant overtime are also plotted. These are also
colored by response class to see if there is a difference between the
classes. We can see from these charts that \textbf{SYSBP} and
\textbf{CIGPDAY} values seem to be higher for individuals that have
hypertension versus those that do not. Therefore, we will consider using
these values over others in the models.

    \hypertarget{figure-1-age-density}{%
\paragraph{Figure 1: AGE Density}\label{figure-1-age-density}}

    \includegraphics{output_12_1.png}

    \hypertarget{figure-2-totchol-density}{%
\paragraph{Figure 2: TOTCHOL Density}\label{figure-2-totchol-density}}

    \includegraphics{output_13_1.png}

    \hypertarget{figure-3-totchol-trajectories}{%
\paragraph{Figure 3: TOTCHOL
Trajectories}\label{figure-3-totchol-trajectories}}

    \includegraphics{output_14_1.png}

    \hypertarget{figure-4-sysbp-density}{%
\paragraph{Figure 4: SYSBP Density}\label{figure-4-sysbp-density}}

    \includegraphics{output_15_1.png}

    \hypertarget{figure-5-sysbp-trajectories}{%
\paragraph{Figure 5: SYSBP
Trajectories}\label{figure-5-sysbp-trajectories}}

    \includegraphics{output_16_1.png}

    \hypertarget{figure-6-diabp-density}{%
\paragraph{Figure 6: DIABP Density}\label{figure-6-diabp-density}}

    \includegraphics{output_17_1.png}

    \hypertarget{figure-7-diabp-trajectories}{%
\paragraph{Figure 7: DIABP
Trajectories}\label{figure-7-diabp-trajectories}}

    \includegraphics{output_18_1.png}

    \hypertarget{figure-8-cigpday-density}{%
\paragraph{Figure 8: CIGPDAY Density}\label{figure-8-cigpday-density}}

    \includegraphics{output_19_1.png}

    \hypertarget{figure-9-cigpday-trajectories}{%
\paragraph{Figure 9: CIGPDAY
Trajectories}\label{figure-9-cigpday-trajectories}}

    \includegraphics{output_20_1.png}

    \hypertarget{figure-10-bmi-density}{%
\paragraph{Figure 10: BMI Density}\label{figure-10-bmi-density}}

    \includegraphics{output_21_1.png}

    \hypertarget{figure-11-bmi-trajectories}{%
\paragraph{Figure 11: BMI
Trajectories}\label{figure-11-bmi-trajectories}}

    \includegraphics{output_22_1.png}

    \hypertarget{figure-12-heartrte-density}{%
\paragraph{Figure 12: HEARTRTE
Density}\label{figure-12-heartrte-density}}

    \includegraphics{output_23_1.png}

    \hypertarget{figure-13-heartrte-trajectories}{%
\paragraph{Figure 13: HEARTRTE
Trajectories}\label{figure-13-heartrte-trajectories}}

    \includegraphics{output_24_1.png}

    \hypertarget{figure-14-glucose-density}{%
\paragraph{Figure 14: Glucose Density}\label{figure-14-glucose-density}}

    \includegraphics{output_25_1.png}

    \hypertarget{figure-15-glucose-trajectories}{%
\paragraph{Figure 15: GLUCOSE
Trajectories}\label{figure-15-glucose-trajectories}}

    \includegraphics{output_26_1.png}

    \hypertarget{figure-16-prevhyp-density}{%
\paragraph{Figure 16: PREVHYP Density}\label{figure-16-prevhyp-density}}

    \includegraphics{output_27_1.png}

    \hypertarget{figure-17-prevhyp-trajectories}{%
\paragraph{Figure 17: PREVHYP
Trajectories}\label{figure-17-prevhyp-trajectories}}

    \includegraphics{output_28_1.png}

    \hypertarget{model-fitting}{%
\subsection{Model Fitting}\label{model-fitting}}

    Four different models were fit, all of which were Generalized Linear
Mixed-Effects models for longitudinal data. As the response is a binary
categorical variable, the binomial family is utilized. For every model,
I also utilized a random intercept. I decided not to explore interaction
terms, for reasons that are apparent later. The first model uses every
variable, while the second uses every except for \textbf{PERIOD}. I
decided to omit \textbf{PERIOD} in the second model, as it is just a
categorical that bins \textbf{TIME} and likely will have a high
correlation with \textbf{TIME} and not much predictive power. It could
be useful in interaction terms, however, I decided not to use any
interaction terms. The third model fit utilized the two continuous
variables with differing trajectory values, \textbf{SYSBP} and
\textbf{CIGPDAY}, as well as the three categoricals with chi squared
tests that had p values below .05, \textbf{CURSMOKE}, \textbf{DIABETES},
and \textbf{EDUC}. The final model fit utilized only the two continuous
variables \textbf{SYSBP} and \textbf{CIGPDAY}. The second model and
first model had comperable AIC and BIC as the best models, with values
of 952.4222 and 1053.373 for the first model and 950.2637 and 1040.588
for the second model. Given the second model has slightly better values
for AIC and BIC and is slightly less complicated, this model is selected
for as the final model. The qqplots for all of these models also show
slighly fatter tails, however, this is ok for the fit. There is no
apparent left or right skew that needs to be adjusted for.

    \hypertarget{figure-18-model-1-qqplot}{%
\paragraph{Figure 18: Model 1 QQplot}\label{figure-18-model-1-qqplot}}

    \includegraphics{output_31_1.png}

    \hypertarget{figure-19-model-2-qqplot}{%
\paragraph{Figure 19: Model 2 QQplot}\label{figure-19-model-2-qqplot}}

    \includegraphics{output_31_1.png}

    \hypertarget{figure-20-model-3-qqplot}{%
\paragraph{Figure 20: Model 3 QQplot}\label{figure-20-model-3-qqplot}}

    \includegraphics{output_35_1.png}

    \hypertarget{figure-21-model-4-qqplot}{%
\paragraph{Figure 21: Model 4 QQplot}\label{figure-21-model-4-qqplot}}

    \includegraphics{output_37_1.png}

    \hypertarget{model-evaluation}{%
\subsection{Model Evaluation}\label{model-evaluation}}

    ROC curves were generated for all models, and the area under these
curves was calculated as well. For the final model, the auroc was .9942,
while it was .9937, .9631, and .964 for the other models respectively.
This is a fantastic fit, and shows why I decided not test interaction
terms.

    \hypertarget{figure-22-roc-curve-final-modelmodel-2}{%
\paragraph{Figure 22: ROC Curve Final Model(model
2)}\label{figure-22-roc-curve-final-modelmodel-2}}

    \includegraphics{output_41_1.png}

    \hypertarget{figure-23-roc-curve-model-1}{%
\paragraph{Figure 23: ROC Curve Model
1}\label{figure-23-roc-curve-model-1}}

    \includegraphics{output_42_1.png}

    \hypertarget{figure-24-roc-curve-model-3}{%
\paragraph{Figure 24: ROC Curve Model
3}\label{figure-24-roc-curve-model-3}}

    \includegraphics{output_43_1.png}

    \hypertarget{figure-25-roc-curve-model-4}{%
\paragraph{Figure 25: ROC Curve Model
4}\label{figure-25-roc-curve-model-4}}

    \includegraphics{output_44_1.png}

    The GAM was the best fit model in the previous data analysis on this
dataset. The auroc was .9258 for the final GAM, which is considerably
worse than all of the models fit in this analysis. We can conclude that
utilizing longitudinal data allows for a better fit model.

    \hypertarget{figure-26-roc-curve-old-model}{%
\paragraph{Figure 26: ROC Curve Old
Model}\label{figure-26-roc-curve-old-model}}

    \includegraphics{../analysis_5/output_42_1.png}


    % Add a bibliography block to the postdoc
    
    
    
    \end{document}
